\chapter{Weektabel}

%\newcommand{\specialcell}[2][c]{%
%	\begin{tabular}[#1]{@{}c@{}}#2\end{tabular}}
In tabel \ref{tabel:wkplan} wordt de planning van het practicum weergegeven.\\
Hierbij is rekening gehouden met eventuele lesuitval in verband met feestdagen.\\~\\
Zorg dat je op schema loopt. Je mag altijd vooruit werken!
\begin{table}[h!]
%	\begin{tabular}{|p{0.8cm}|c{1.8cm}|l|l|}
	\begin{tabular}{ | m{1.5cm} | m{7cm} |p{2.6cm}|p{3cm}| } 
%	\begin{tabularx}{\textwidth}{|X|l|X|}
		\hline
lesweek &Opdracht & Practicum dictaat &Herkansing opdracht \\ \hline
	10	& Introductie in Visual Studio Code en de klasse Led. & Hoofdstuk \ref{chap:klas} tot \ref{sec:startDDD}& Week 12\\ \hline
	11	&   LED objecten, zichtbaar in de DDD debugger &  Hoofdstuk \ref{sec:startDDD}&Week 13 \\ \hline
	12\&13	&  Afgeleide klassen en objecten & Hoofdstuk \ref{chap:inh} &Week 14\\ \hline
	14	&Compositie en aggregatie    & Hoofdstuk \ref{ch:hfstCompAgg}&Week 15 \\ \hline
	15\&16	& Exceptions   & Hoofdstuk \ref{chap:Excep}& Week 17 \\ \hline
	17	&Unit testing  & Hoofdstuk \ref{chap:unittst} & Week 18\\ \hline	
	18 &Uitloop& &\\ \hline
	\end{tabular}  
    \caption{Planning van het practicum.}
    \label{tabel:wkplan}
\end{table}

