\chapter{Introductie in unit testing} \label{chap:unittst}

Deze opdracht bestaat uit twee onderdelen die beide afgetekend dienen te worden.
Het eerste onderdeel is een introductie in een unit test. Het tweede onderdeel is een het toepassen van een unit test op een klasse uit de vorige opgave.

\section{Deel 1, introductie}

Unit testen zijn testen die gedaan worden op code niveau en in eerste instantie door de programmeur zelf. 
Er zijn diverse frameworks in omloop om de unit testen uit te kunnen uitvoeren.
Een relatief veel gebruikt framework voor de C++ omgeving is het \href {https://github.com/google/googletest}{google framework}.         
Bij deze opgave wordt gebruik gemaakt van het \href{https://cpputest.github.io/} {CppUTest} framework. 
Dit is een relatief eenvoudig framework en kan geïnstalleerd worden via het commando:\\
\texttt{sudo apt-get install cpputest}.

Een voorbeeld van het CppUTest framework wordt gegeven in listing \ref{lst:unitTst1} Hierbij worden twee testen uitgevoerd die allebei falen.
\begin{lstlisting}[caption= Een eenvoudige unittest. ,frame=trbl,firstnumber=1,numbers=left,label={lst:unitTst1}]{Name}
#include <CppUTest/CommandLineTestRunner.h>
#include <CppUTest/TestHarness.h>

/* The definition of a TEST_GROUP, the name is sample */
TEST_GROUP(FirstTestGroup)
{};

/* The definition of a belonging to the TEST_GROUP TEST, the name is ret_int_success */
TEST(FirstTestGroup, FirstTest)
{
	FAIL("Fail me!");
}

/* The definition of a belonging to the TEST_GROUP TEST, the name is ret_int_failed */
TEST(FirstTestGroup, SecondTest)
{
	STRCMP_EQUAL("hello", "world");
	
}

int main(int argc, char *argv[])
{
	CommandLineTestRunner::RunAllTests(argc, argv);
	return 0;
}
\end{lstlisting}
Download het \textit{test1.c}  van git:\\
{\small \texttt{git clone } \verb|--| \texttt{branch unittest https://github.com/JohnVi-hhs/oop.git}}\\
compileer het programma \texttt{g++ test1.c -lCppUTest -o test} en 
run het programma   \texttt{./test}

Het resultaat van test1 is dat de beide testen   \textit{TEST(FirstTestGroup, FirstTest) } en \textit{TEST(FirstTestGroup, SecondTest) }
falen.

Indien we naar de code kijken, zien we dat deze uit één testgroep en twee  losse testen bestaan.
De rede dat beide testen falen heeft te maken met de keywords FAIL en STRCMP. 
De test aan de hand van het keyword FAIL, faalt altijd. De test aan de hand van het keyword STRCMP vergelijkt 2 strings,
aangezien de string "hello" niet gelijk is aan "world" faalt ook deze test. Andere keywords (ook wel assertions genoemd)
zijn terug te vinden in de \href{https://cpputest.github.io/manual.html} {manual}.

\paragraph{Opdracht}

\begin{enumerate}[label=(\Alph*)]
	\item 
Zorg ervoor dat één van beide testen \textbf{niet} meer faalt, bedenk hierna zelf een 3e test, maak hierbij gebruik van een testkeyword (assertion) die nog niet in dit voorbeeld gebruikt wordt.

\item 
In de TEST\_GROUP kunnen ook nog de functies \texttt{void setup();} en \texttt{void teardown();} gedefinieerd worden. Dit is te zien in listing \ref{lst:unitTstGroup}.
\begin{lstlisting}[caption= De TEST\_GROUP met de \textit{setup} en \textit{teardown} functies. ,frame=trbl,firstnumber=1,numbers=left,label={lst:unitTstGroup}]{Name}
#include <iostream>
using namespace std;

TEST_GROUP(FirstTestGroup)
{
	void setup() 
	{
		cout<<"voor de test"<<endl;
	}    
	void teardown()
	{
		cout<<"Na de test"<<endl;
	}
};
\end{lstlisting}
De functie setup wordt aangeroepen voordat een test wordt aangeroepen, de functie teardown wordt aangeroepen nadat een test is uitgevoerd.

Voeg de functies \texttt{setup} en \texttt{teardown} toe aan de test en voer de test opnieuw uit.

\item 
De testomgeving zoals hierboven beschreven is, is niet echt spannend. Het zou beter zijn wanneer een eigen klasse getest kan worden. In listing \ref{lst:frontBh} en \ref{lst:frontBc} wordt de klasse FrontBoek weergegeven.

\noindent\hspace{-1cm}\begin{minipage}{.43\textwidth}
\begin{lstlisting}[caption=FrontBoek declaratie file(.h),frame=tlrb,label={lst:frontBh}]{Name}
class FrontBoek {
  public:
    FrontBoek(string,string);
    virtual ~FrontBoek();
	string naamSchrijver()const;
	string naamTitel()const;
	void verhoogSchrijver();
  private:
    string schrijver;
	string titel;
};
	\end{lstlisting}
\end{minipage}\hfill
\hspace{0.5cm}\begin{minipage}{.55\textwidth}
	\begin{lstlisting}[caption=FrontBoek implementatie file(.cpp),frame=tlrb,label={lst:frontBc}]{Name}
FrontBoek::FrontBoek(string n,string t):schrijver(n),titel(t) {
}
FrontBoek::~FrontBoek() {
}
string FrontBoek::naamSchrijver()const {
	return schrijver;
}
string FrontBoek::naamTitel()const {
	return titel;
}
void FrontBoek::verhoogSchrijver() {
	schrijver +=1;
}
	\end{lstlisting}
\end{minipage}
Het is een eenvoudige klasse met de naam van de schrijver en de titel als private variabele. Verder heeft iemand nog een functie 
\textit{void verhoogSchrijver();} verzonnen, waarmee een indicatie verkregen kan worden hoeveel personen aan het boek hebben meegewerkt.
Indien de klasse met het framework getest gaat worden, zou deze eruit kunnen zien zoals in listing \ref{lst:unitTstBk} wordt weergegeven.
\begin{lstlisting}[caption= De Frontboek test. ,frame=trbl,firstnumber=1,numbers=left,label={lst:unitTstBk}]{Name}
#include <CppUTest/CommandLineTestRunner.h>
#include <CppUTest/TestHarness.h>
#include <iostream>
using namespace std;

#include "FrontBoek.h"

TEST_GROUP(boekengroep)
{
	void setup()  {
		cout<<"De boekentest"<<endl;
	}
	void teardown() {
		cout<<"Einde boekentest"<<endl;
	}
};
TEST(boekengroep, front)
{
	FrontBoek fb1("L&B","lab");
	STRCMP_EQUAL("L&B",fb1.naamSchrijver().c_str());
}

int main(int argc, char *argv[])
{
	CommandLineTestRunner::RunAllTests(argc, argv);
	return 0;
}
\end{lstlisting}
Compileer Listing \ref{lst:unitTstBk} (test3.c). Houd er bij het compileren rekening mee dat de klasse FrontBoek ook gecompileerd wordt. Dit kan gedaan worden door de file FrontBoek.cpp mee te nemen in de compileer regel:
\texttt{g++ FrontBoek.cpp test3.c -lCppUTest -o test3} en run het programma \texttt{./test3}
\item Maak een eigen test waarmee de functie verhoogschrijver van de klasse FrontBoek getest wordt en voer de test uit.

\end{enumerate}

\section{Deel2, een eigen test}

\begin{enumerate}[label=(\Alph*)]
	\item 
Maak eerst een unit test voor de klasse RGBLed en vervolgens voor de klasse Logled. Bedenk dat er kritisch getest moet worden.

\item Laat het resultaat aftekenen en plaats de volledig test in je portfolio.
\end{enumerate}