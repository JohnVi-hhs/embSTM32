\section{De klasse Tijdsduur.}

We willen een ADT (Abstract Data Type) , ook wel zelfgedefinieerd datatype genoemd, maken waarin een tijdsduur in minuten en seconden kan worden opgeslagen. De totale tijd in seconden kan ook worden opgevraagd. 
\begin{figure}[h!]
	\captionsetup{justification=centering}
	\includegraphics[width=0.28 \linewidth]{figuren/tijdsduur}
	\centering
	\caption{weergave van de klasse Tijdsduur. }
	\label{fig:tijdsduurKlas}
\end{figure}
We noemen dit zelf gedefinieerde datatype \texttt{Tijdsduur}. Figuur \ref{fig:tijdsduurKlas} geeft de klasse weer van Tijdsduur.
Het ADT \texttt{Tijdsduur} kan in C++ als volgt gedeclareerd worden in de header file (tijdsduur.h):

\begin{lstlisting}[caption= de headerfile van de klasse \texttt{Tijdsduur},label={lst:tijdsdHeader},numbers=none]		
	#ifndef TIJDSDUUR_H
	#define TIJDSDUUR_H
	
	// De declaratie van de ADT Tijdsduur:
	class Tijdsduur {
		public:
		//...
		void eraf(Tijdsduur t);
		//...
		
		private:
		int sec;
		//...
	};
	#endif // TIJDSDUUR_H
\end{lstlisting}

De implementatie (\texttt{tijdsduur.cpp}) ziet er tot nu toe als volgt uit:
\begin{lstlisting}[caption= de implementatiefile van de klasse \texttt{Tijdsduur},label={lst:tijdsdImpl},numbers=none]
	#include <iostream>
	#include "tijdsduur.h"
	#include <iomanip>
	using namespace std;
	
	// De definities van de memberfunctie van de ADT Tijdsduur, oftewel: de implementatie van de ADT Tijdsduur:
	void Tijdsduur::eraf(Tijdsduur t) {
		sec-=t.sec;
		//...
	}
\end{lstlisting}

Het hoofdprogramma (\texttt{testTijdsduur.cpp}) ziet er als volgt uit:

\begin{lstlisting}[caption= de implementatiefile van het hoofdprogramma ,label={lst:tijdsdMainprog},numbers=none]
	#include <iostream> // nodig voor cout (schrijven naar scherm)
	#include <iomanip> // nodig voor setw (veldbreedte definieren )
	#include "tijdsduur.h"
	using namespace std;
	
	int main() {
		Tijdsduur t1(3,50); // t1 is 3 minuten en 50 seconden
		cout<<"t1 = "; t1.print(); cout<<endl;
		const Tijdsduur kw(15); // kw is 15 seconden
		cout<<"kw = "; kw.print(); cout<<endl;
		t1.erbij(kw); // Tel kw bij t1 op
		cout<<"t1 = "; t1.print(); cout<<endl;
		Tijdsduur t2(t1); // t2 is een kopie van t1
		t2.eraf(kw); // Trek kw van t2 af
		cout<<"t2 = "; t2.print(); cout<<endl;
		t2.maal(7); // Vermenigvuldig t2 met 7
		cout<<"t2 = "; t2.print(); cout<<endl;
		Tijdsduur t3(3,-122); // t3 is 3 minuten minus 122 seconden
		cout<<"t3 = "; t3.print(); cout<<endl;
		t3.eraf(t2); 
		cout<<"t3 = "; t3.print(); cout<<endl;
		Tijdsduur t4(3,122); // t4 is 3 minuten plus 122 seconden
		cout<<"t4 = "; t4.print(); cout<<endl;
		cout<<"het totaal aantal seconde van t4 = "<<t4.deTimerTijd()<<endl;
		return 0;
	}	
\end{lstlisting}
De uitvoer moet dan zijn:

\begin{tabular}{ l l l }
	t1= & 3 minuten en & 50 seconden \\ 
	kw	=& &15	seconden \\  
	t1	=&	4	minuten en&	5	seconden\\
	t2	=&	3	minuten en	&50	seconden\\
	t2	=&	26	minuten en	&50	seconden\\
	t3	=&			&58	seconden\\
	t3	=&			&0	seconden\\
	t4	=&	5	minuten en	&2	seconden
	
\end{tabular}

\paragraph{Opdracht}
\begin{enumerate}[label=\alph*]
	\item De code van de implementatie van de klasse tijdsduur, zoals hierboven vermeld is, is verre van compleet.
	Download  opg13.zip van Brightspace of clone deze:\\ 
	{\small \texttt{git clone } \verb|--| \texttt{branch logled https://github.com/JohnVi-hhs/oop.git}}
	
	Vul de declaratie en de implementatie van de ADT genaamd  Tijdsduur verder in. Zorg ervoor dat het hoofdprogramma (\texttt{testTijdsduur.cpp}) zonder warnings te compileren is en de gewenste uitvoer produceert. Zoek (indien nodig) inspiratie bij de in de les behandelde \texttt{\textbf{class} Breuk}. Tip: Zorg ervoor dat de opgeslagen seconden altijd \textgreater =0 en \textless 60 zijn.
	\item Voer zelf nog een aantal testen met Tijdsduur uit, bijvoorbeeld:
	\begin{itemize}
		\item Het testen op een negatieve tijd (een negatieve tijd bestaat niet).
		\item Het gebruik van de methode\texttt{ int deTimerTijd()}.
	\end{itemize} 
	\item Laat de opdracht aftekenen.
\end{enumerate}
