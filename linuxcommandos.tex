\hypertarget{LinuxTipsTrics}{}
\chapter{Tips\&Tricks voor Linux omgeving} \label{app:linux}
\section{Tips} \label{app:tips}
\textbf{\textit{Je kunt de tips en trucs toepassen bij de commando's in \ref{app:commands}}}

\begin{itemize}
	\item 
	Linux is in tegenstelling tot Windows ‘case-sensitive’. Het maakt dus verschil of je hoofdletters of kleine letters gebruikt. Voorbeeld: Als je map \texttt{Documents} heet, dan werkt \texttt{cd Documents} wel en \texttt{cd documents} niet.
	\item 
	Tip bij het invoeren van map- en bestandsnamen: tik de eerste 3 letters in en druk de Tab toets in, Linux vult dan automatisch de rest van de naam in. Voordeel: je hoeft minder te tikken en het is gelijk een foutcontrole.
	\item 
	Op en neer pijltjestoetsen in de terminal: door vorige commando's heen gaan.
	\item
	Nog een verschil met Windows; Windows werkt met bestandsextensies: *.exe is een uitvoerbaar (executable) betand. In Linux doet de extensie er niet toe, maar zet je met \texttt{chmod +x} het 'executable' bitje aan, een attribuut op het bestand.
	\item  
	Hulp vragen bij commando’s: \texttt{<\textit{commando}> --help}. Voorbeeld: \texttt{ls  --help}
	Uitgebreidere informatie: \texttt{man <\textit{commando}>}. Voorbeeld: \texttt{man ls}\newline
	Voor de bedieningstoetsen van \texttt{man} druk ‘h’.
	\item 
	\href{https://phoenixnap.com/kb/linux-sudo-command}{Uitleg en gebruik van het sudo commando}  ('\textbf{Su}per \textbf{U}ser \textbf{Do}'); admin rechten.
	\item
	Zoeken in het bestandssysteem: \texttt{sudo find / -name \textit{<bestandsnaam>}}
	zoekt naar \texttt{\textit{<bestandsnaam>}} vanaf de root '/' in het gehele bestandssysteem.
	\item 
	\href{https://linuxconfig.org/filesystem-basics}{Uitleg van- en navigeren door het bestandssysteem.}
	\item
	Wil je meer uitleg, \href{https://duckduckgo.com/?q=linux+command+cheatsheet}{zoek dan naar 'Linux Command Cheatsheet'}. \newline
	Ik vind \href{https://linuxconfig.org/linux-commands-cheat-sheet}{deze uitleg} wel prettig.
\end{itemize}

\break
\section{Linux commando's (zie ook tips in \ref{app:tips})}\label{app:commands}

\begin{table}[ht] %[h!]
	\begin{tabular}{|l|l|}
		\hline
		%\multicolumn{2}{|l|}{Linux commando's}    \\ \hline
		Commando & Uitleg \\ \hline
		\texttt{ls \textbf{-l}} & \textbf{L}ange directory listing - Opvragen van bestanden in een directory  \\ \hline
		\texttt{cd} & change directory; \texttt{cd ..} betekent 1 niveau omhoog naar de root '/' \\ \hline
	    \texttt{cp} & copy \\ \hline
	    \texttt{rm} & remove - Verwijderen van bestanden en mappen. \\ \hline
	    \texttt{mv} & move - Verplaatsen van bestanden en mappen. \\ \hline
	    \texttt{cat} & bestand bekijken \\ \hline
	    \texttt{pwd} & In welke directoy sta ik? ('present working directory') \\ \hline
	    \texttt{mkdir} & make directory \\ \hline
		\texttt{rmdir} & remove (lege) directory \\ \hline
	    \texttt{history} & Laat lijst zien van laatst uitgevoerde commando's \\ \hline
	    \texttt{sudo} & Zet dit vóór je commando om het met admin rechten uit te voeren. \\ \hline
	    \texttt{chmod} & (met sudo) lees/schrijfrechten wijzigen, +x maakt bestand executable. \\ \hline
	    \texttt{chown} & (met sudo) eigenaar van bestand / map wijzigen \\ \hline
	\end{tabular}
\end{table}

\break
\section{GPIO poorten in Linux}\label{app:gpiolinux}
In Linux wordt alles gerepresenteerd als een bestand. Of het nu een volledige disk of een device is, je kunt het als een bestand openen en lezen en schrijven. Dit geldt ook voor GPIO pinnen (‘General Purpose Input/Output’ pinnen). Deze pinnen gebruik je op dezelfde manier als de pinnen van een microcontroller (denk aan een Arduino Uno, of een Microbit). Je kunt pinnen als uitgang zetten en hoog of laag maken, of als ingang instellen en uitlezen of een pin hoog of laag is (en nog meer, bijvoorbeeld pull-up en pull-down en reageren op events, maar dat gaan we niet behandelen).\newline
Omdat pinnen gerepresenteerd worden door bestanden, geldt ook dat er rechten op gezet worden in het bestandssysteem. Meestal kan een standaard gebruiker (zoal ‘rock’) er niet bij. Alleen ‘root’ kan er standaard bij. Voor het practicum hebben we de rechten zo ingesteld dat gebruiker ‘rock’ standaard gebruik kan maken van de op de breakout board beschikbare pinnen.\newline
Voorbeeld van de handelingen om een pin in te stellen en uit te lezen of aan te sturen, het hekje ‘\#’ aan het begin van de regel geeft aan dat het commentaar (uitleg) is:

\begin{lstlisting}[language=bash]
# GPIO pin 135 (de rode led) aanzetten:
echo 135 > /sys/class/gpio/export
# GPIO pin 135 (de rode led) als uitgang instellen:
echo out > /sys/class/gpio/gpio135/direction
# GPIO pin 135 (de rode led) hoog maken:
echo 1 > /sys/class/gpio/gpio135/value
# 1 seconde wachten
sleep 1
# GPIO pin 135 (de rode led) laag maken:
echo 0 > /sys/class/gpio/gpio135/value
\end{lstlisting}
	
Je kunt ook de hele bestandstructuur van de pin bekijken:
\begin{lstlisting}[language=bash]
rock@rockpi-4b:~$ ls -l /sys/class/gpio/gpio135/
total 0
-rwSrwSr-- 1 root gpio 4096 Mar 12 18:49 active_low
lrwxrwxrwx 1 root gpio    0 Mar 12 18:49 device -> ../../../pinctrl
-rwSrwSr-- 1 root gpio 4096 Mar 12 18:49 direction
-rwSrwSr-- 1 root gpio 4096 Mar 12 18:49 edge
drwsrwsr-x 2 root gpio    0 Mar 12 18:49 power
lrwxrwxrwx 1 root gpio    0 Mar 12 18:49 subsystem -> ../../../../../class/gpio
-rwSrwSr-- 1 root gpio 4096 Mar 12 18:49 uevent
-rwSrwSr-- 1 root gpio 4096 Mar 13 10:55 value
\end{lstlisting}
	
En een bepaalde status uitlezen (in dit geval of de pin als ingang of uitgang is ingesteld):
\begin{lstlisting}[language=bash]
rock@rockpi-4b:~$ cat /sys/class/gpio/gpio135/direction
out
rock@rockpi-4b:~$
\end{lstlisting}
	
Tot zover de uitleg. Ik laat GPIO pin als ingang en gebruik met PWM voor nu achterwege. Als je daar meer van wilt weten, kijk dan even naar het zelftest scriptje: \texttt{sudo cat /usr/local/bin/pwmtest.sh} 

