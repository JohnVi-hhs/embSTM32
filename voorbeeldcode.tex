\chapter{Voorbeeld code}
\section{De klasse Aanwezigheid}\label{cpt:klAanw}
\begin{lstlisting}[caption=Het werken met de klasse Aanwezigheid, frame=tlrb,label={lst:vbtstAanwezig}]{Name}
#include <iostream>
#include "Aanwezigheid.h"
using namespace std;

int main() {
	const Aanwezigheid lokaalD2136(11);
	Aanwezigheid lokaalD2001;
	lokaalD2001.naarBinnen();
	lokaalD2001.naarBinnen();
	lokaalD2001.verlaat();
	cout<<"Het aantal in lokaal D2136="<<lokaalD2136.zijnAanwezig()<<endl;
	cout<<"Het aantal in lokaal D2001="<<lokaalD2001.zijnAanwezig()<<endl;
	return 0;
}
\end{lstlisting}
\noindent\hspace{-1cm}\begin{minipage}{.43\textwidth}
	\begin{lstlisting}[caption=Aanwezigheid declaratie file(.h),frame=tlrb,label={lst:vbAanwezigH}]{Name}
class Aanwezigheid
{
	public:
	  Aanwezigheid();
	  Aanwezigheid(int);
	  void naarBinnen();
	  void verlaat();
	  int zijnAanwezig()const;
	
	private:
	  int aantal;
};
	\end{lstlisting}
\end{minipage}\hfill
\hspace{0.7cm}\begin{minipage}{.70\textwidth}
\begin{lstlisting}[caption=Aanwezigheid implementatie file(.cpp),frame=tlrb,label={lst:vbAanwezigCPP}]{Name}
#include "Aanwezigheid.h"

Aanwezigheid::Aanwezigheid():aantal(0) {
}
Aanwezigheid::Aanwezigheid(int x):aantal(x) {
}
void Aanwezigheid::naarBinnen() {
	aantal++;
}
void Aanwezigheid::verlaat() {
	aantal--;
}
int Aanwezigheid::zijnAanwezig()const {
	return aantal;
}
\end{lstlisting}
\end{minipage}
\newpage
\section{De klasse Breuk}\label{cpt:klBreuk}
\vspace{0.1cm}\begin{lstlisting}[caption=Het werken met de klasse Aanwezigheid, frame=tlrb,label={lst:vbtstAanwezig}]{Name}
#include <iostream>
using namespace std;
#include "Breuk.h"
int main() {
	Breuk a(1,2);
	Breuk b;
	Breuk c(2);  
	a.plus(c);
	cout<<b.teller()<<" "<<b.noemer();
	return 0;
}
\end{lstlisting}
\noindent\hspace{-1cm}\begin{minipage}{.43\textwidth}
	\begin{lstlisting}[caption=Breuk declaratie file(.h),frame=tlrb,label={lst:vbBreukH}]{Name}
class Breuk {
	public:
	Breuk();
	Breuk(int);  
	Breuk(int,int);               
	void drukaf();   
	int teller()const;
	int noemer()const;
	void plus(const Breuk& b);
	private:                       
	int boven;                 
	int onder;                
	void normaliseer();        
};
	\end{lstlisting}
\end{minipage}\hfill
\hspace{0.7cm}\begin{minipage}{.70\textwidth}
	\begin{lstlisting}[caption=Breuk implementatie file file(.cpp),frame=tlrb,label={lst:vbBreukCPP}]{Name}
#include "Breuk.h"
#include <iostream>
using namespace std;
Breuk::Breuk():boven(0),onder(1) {
}
Breuk::Breuk(int t):boven(t),onder(1) {
}
Breuk::Breuk(int t,int n):boven(t),onder(n) {
}
void Breuk::drukaf() {
	cout << boven << "/" << onder << endl;
}
void Breuk::plus(const Breuk& b) {
	boven = boven * b.onder + onder * b.boven;
	onder *= b.onder;
	normaliseer();
}
int Breuk::teller()const {
	return boven;
}
int Breuk::noemer()const {
	return onder;
}
unsigned int ggd(unsigned int n, unsigned int m) {
	if (n == 0) return m;
	if (m == 0) return n;
	while (m != n)
		if (n > m) n -= m;
    	else m -= n;
	return n;
	}
void Breuk::normaliseer() {
	if (onder < 0) {
		onder =- onder;
		boven =- boven;
	}
	int d(ggd(boven < 0 ? -boven : boven, onder));
	boven /= d;
	onder /= d;
}
\end{lstlisting}
\end{minipage}

\section{De verschillende honden klassen}\label{cpt:klHonden}
\vspace{0.1cm}\begin{lstlisting}[caption=Het werken met de klassen Hond en afgeleide, frame=tlrb,label={lst:vbtstAanwezig}]{Name}
#include <iostream>
using namespace std;
#include "Tekkel.h"
#include "SintBernard.h"

void doeJeWerk(Hond& h) {
	h.haalKrant();
	// ...
}
int main() {
	
	Tekkel tk("Rocky");
	SintBernard sb("Boris",2);
	Hond& refHd1(tk);
	doeJeWerk(sb);
	doeJeWerk(tk);
	return 0;
}
\end{lstlisting}
\newpage
\subsection{De klasse Hond}
De code van de header-(.h) en de implementatiefile(.cpp)\\ 

\noindent\hspace{-1.1cm}\begin{minipage}{.58\textwidth}
\begin{lstlisting}[caption=Hond declaratie file(.h),frame=tlrb,label={lst:vbHondH}]{Name}
#include<string>

using namespace std;

class Hond {
   private:
	   string naam;
	
   public:
       Hond(string);
       Hond();
       virtual ~Hond();
	   void haalKrant();
 /*pure virtual functie void blaf()
   Hond heeft geen eigen blaf functie*/
	   virtual void blaf()=0; 
	
 /*
  indien geen pure virtual functie
  Hond moet een eigen blaf functie hebben
  (virtual) void blaf(); 
  */
	
	   string name()const;
};
\end{lstlisting}
\end{minipage}\hfill
\hspace{0.9cm}\begin{minipage}{.58\textwidth}
\begin{lstlisting}[caption=Hond implementatie file file(.cpp),frame=tlrb,label={lst:vbHondCPP}]{Name}
#include "Hond.h"
#include "Tekkel.h"
#include <iostream>

using namespace std;


//constructoren
Hond::Hond()
{
	cout<<"ik ben een Hond"<<endl;
}
Hond::Hond(string s): naam(s)
{
	cout<<"ik ben een Hond"<<endl;
}

//destructor
Hond::~Hond()
{
	cout<<"ik was een hond"<<endl;
}

string Hond::name()const {
	return naam;
}


void Hond::haalKrant() {
	// ...
	blaf();
}

/*
De functie blaf indien in de .h file
niet pure virtual gedeclareerd is.

void Hond::blaf() {
	cout<<naam<<" blaf blaf"<<endl;
}

*/

\end{lstlisting}
\end{minipage}

\newpage
\subsection{De klasse Tekkel}
De code van de header-(.h) en de implementatiefile(.cpp)\\ 

\noindent\hspace{-1cm}\begin{minipage}{.43\textwidth}
\begin{lstlisting}[caption=Tekkel declaratie file(.h),frame=tlrb,label={lst:vbTekkelH}]{Name}
#include <string>
using namespace std;

#include "Hond.h"

class Tekkel :public Hond{
	public:
      Tekkel(string);
	  ~Tekkel();
	  void blaf();
};		
\end{lstlisting}
\end{minipage}\hfill
\hspace{0.7cm}\begin{minipage}{.60\textwidth}
\begin{lstlisting}[caption=Tekkel implementatie file file(.cpp),frame=tlrb,label={lst:vbTekkelCPP}]{Name}
#include "Tekkel.h"
#include <iostream>
using namespace std;


Tekkel::Tekkel(string s):Hond(s) {
	cout<<"ik ben een tekkel"<<endl;
}
Tekkel::~Tekkel() {
	cout<<"ik was een Tekkel"<<endl;
}
void Tekkel::blaf() {
	cout<< "kef kef"<<endl;
}		
\end{lstlisting}
\end{minipage}

\subsection{De klasse SintBernard}
De code van de header-(.h) en de implementatiefile(.cpp)\\ 

\noindent\hspace{-1cm}\begin{minipage}{.43\textwidth}
\begin{lstlisting}[caption=SintBernard declaratie file(.h),frame=tlrb,label={lst:vbSintBernardH}]{Name}
#include "Hond.h"

class SintBernard :public Hond{
	
   public:
      SintBernard(int);
      SintBernard(string);
      SintBernard(string,int);
      ~SintBernard();
      void blaf();
	
	private:
	  int whiskeyvat;
};		
\end{lstlisting}
\end{minipage}\hfill
\hspace{0.7cm}\begin{minipage}{.70\textwidth}
	\begin{lstlisting}[caption=SintBernard implementatie file file(.cpp),frame=tlrb,label={lst:vbSintBernardCPP}]{Name}
#include "SintBernard.h"
#include <iostream>

using namespace std;

SintBernard::SintBernard(int v):whiskeyvat(v)
{
	cout<<"Ik ben een SintBernard"<<endl;
}
SintBernard::SintBernard(string n,int v):Hond(n),whiskeyvat(v)
{
	cout<<"Ik ben een SintBernard"<<endl;
}
SintBernard::SintBernard(string nm):Hond(nm),whiskeyvat(0)
{
	cout<<"Ik ben een SintBernard"<<endl;
}
SintBernard::~SintBernard()
{
	cout<<"Ik was een SintBernard "<<endl;
}
void SintBernard::blaf() {
	cout<<"WOEF WOEF"<<endl;
}		
	\end{lstlisting}
\end{minipage}
